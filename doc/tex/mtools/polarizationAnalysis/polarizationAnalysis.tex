\clearpage

\section{Polarization Analysis Signals}

\begin{tcolorbox}	
	\begin{tabular}{p{2.75cm} p{0.2cm} p{10.5cm}} 	
		\textbf{Students Name}  &:& Mariana Ramos\\
		\textbf{Goal}           &:& Analyse simulation Photon Stream signals into Stokes space using plot in Poincare sphere and Autocorrelation calculation and plot. \\\\
		\textbf{Version}        &:& jonesToStokes\_20180614.m 
	\end{tabular}
\end{tcolorbox}


This section shows some matlab functions to analyse photon stream signals from simulation.

\subsection{jonesToStokes.m}

\subsection*{Structure of a function}

[] = jonesToStokes(fname,deltaP, filename,NumberOfSamples);

\subsection*{Inputs}

\indent

\textbf{fname}: Input filename of the signal (*.sgn) you want to convert. It must be a photon stream signal (Type: PhotonStreamXY).
\bigskip

\textbf{deltaP}: Value of delta P (polarization linewidth which is a parameter that depends on optical fibre installation) used to perform the simulation.
\bigskip

\textbf{filename}: Input filename which contains the stokes parameters resulted from simulation (*.txt).
\bigskip

\textbf{NumberOfSamples}: Number of Samples to plot in Poincare sphere.


\subsection*{Outputs}
Two graphics are plotted from the photon stream input signal:
\bigskip

\textbf{Histogram of parameters $\vec{\alpha}$}: Three plots of each $\alpha_i$ which were used as input parameters of the random matrix used to model the SOP drift block.
\bigskip

\textbf{Poincare sphere}: A plot of the time evolution of the photon stream into the Poincare sphere.



\subsection*{Functional Description}

This matlab function converts the input signal (*.sgn) which is represented in Jones space in a signal represented in Stokes Space which allows us to plot the signal time evolution in Poincare sphere. Furthermore, the $\vec{\alpha}$ parameters obtained from simulation are also plotted in order to be sure that they were generated correctly by following a normal distribution with mean $0$ and a standard deviation depending on these parameters values.

\begin{tcolorbox}	
	\begin{tabular}{p{2.75cm} p{0.2cm} p{10.5cm}} 	
		\textbf{Students Name}  &:& Mariana Ramos\\
		\textbf{Version}        &:& ACF\_20180614.m 
	\end{tabular}
\end{tcolorbox}
\subsection{ACF.m}

\subsection*{Structure of a function}
[] = ACF(fname,deltaP,N)

\subsection*{Inputs}
\indent

\textbf{fname}: Input filename of the signal (*.sgn) you want to convert. It must be a photon stream signal (Type: PhotonStreamXY).
\bigskip

\textbf{deltaP}: Value of delta P (polarization linewidth which is a parameter that depends on optical fibre installation) used to perform the simulation.
\bigskip

\textbf{N}: Number of Samples used to plot the function of autocorrelation.


\subsection*{Outputs}
This matlab function has two outputs:
\bigskip

\textbf{.txt file with values of the autocorrelation calculated with data from simulation}: A .txt file with the autocorrelation of the photon stream signal which inputs the function for $N$ samples which is also an input of the function-.
\bigskip

\textbf{ACF plot}: A plot with numerical ACF calculated from simulation data and with the theoretical ACF calculated based on the value of $\Delta p$ inserted as an input of the function.


\subsection*{Functional Description}
This matlab function calculates the autocorrelation in time domain of the photon stream input signal as well as the theoretical autocorrelation based on the value of $\Delta p$ inserted as an input of the function which must be the same used in simulation. This function outputs a txt file with the numerical ACF and plots a graphic with both theoretical and numerical autocorrelation.

\begin{tcolorbox}	
	\begin{tabular}{p{2.75cm} p{0.2cm} p{10.5cm}} 	
		\textbf{Students Name}  &:& Mariana Ramos\\
		\textbf{Version}      &:& plotPhotonStream\_20180102.m
	\end{tabular}
\end{tcolorbox}
\subsection{plotPhotonStream\_20180102.m}

\subsection*{Structure of a function}
[mag\_x, mag\_y, phase\_dif, h] = plotPhotonStream\_20180102(fname, opt, h)

\subsection*{Inputs}
\indent

\textbf{fname}: Input filename of the signal (*.sgn) you want to convert. It must be a photon stream signal (Type: PhotonStreamXY).
\bigskip

\textbf{opt}: If opt equals 0, or no opt, we plot the absolute value of X and Y and the phase difference.If opt equals 1, we plot the amplitude of X and Y, this is only possible when X and Y are real values.
\bigskip

\textbf{h}: Number of the figure to plot.


\subsection*{Outputs}
This matlab function has four outputs:
\bigskip

\textbf{mag\_x}: Magnitude of the complex number X.
\bigskip

\textbf{mag\_y}: Magnitude of the complex number Y.
\bigskip

\textbf{phase\_dif}: Difference phase between the two complex numbers X and Y.
\bigskip

\textbf{h}: A plot of magnitude and phase difference of the two complex numbers depending on the choice of the input parameter \textbf{opt}.


\subsection*{Functional Description}
This matlab function plots the magnitude of the two components of the input photon stream signal (*.sgn) and the phase difference between both components. This function allows us to visualize the photon stream signal over time.

\begin{tcolorbox}	
	\begin{tabular}{p{2.75cm} p{0.2cm} p{10.5cm}} 	
		\textbf{Students Name}  &:& Mariana Ramos\\
		\textbf{Version}        &:& plot\_sphere.m
	\end{tabular}
\end{tcolorbox}
\subsection{plot\_sphere.m}

This function accepts the three stokes parameters $S_1$, $S_2$ and $S_3$. This function supports the other functions in this section allowing the plot of these parameters in Poincare sphere.


%\end{document}
