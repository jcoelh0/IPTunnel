\clearpage

\section{IP Tunnel}

\begin{tcolorbox}	
	\begin{tabular}{p{2.75cm} p{0.2cm} p{10.5cm}} 	
		\textbf{Header File}   &:& ip$\_$tunnel$\_*$.h \\
		\textbf{Source File}   &:& ip$\_$tunnel$\_*$.cpp \\
        \textbf{Version}       &:& 20180815 (Jo\~ao Coelho) \\
	\end{tabular}
\end{tcolorbox}

This block accepts one input signal and it transmits it. This block, on the other machine, accepts the signal and outputs it. It takes samples out of the buffer until the buffer is empty. It has the option of displaying the number of samples still available.

\subsection*{Input Parameters}

\begin{table}[h]
	\centering
	\begin{tabular}{|c|c|p{30mm}|c|ccp{60mm}}
		\cline{1-4}
		\textbf{Parameter} & \textbf{Type} & \textbf{Values} &   \textbf{Default}& \\ \cline{1-4}
		numberOfSamples & long int & any & $-1$ \\ \cline{1-4}
        displayNumberOfSamples & bool & true/false & true \\ \cline{1-4}
	numberOfTrials & int & any & $5$ \\ \cline{1-4}
	numberOfRepetions & int & any & $3$ \\ \cline{1-4}
	\end{tabular}
	\caption{Sink input parameters}
	\label{table:ipt_in_par}
\end{table}

\subsection*{State Variables}

\begin{table}[h]
	\centering
	\begin{tabular}{|c|c|p{30mm}|c|ccp{60mm}}
		\cline{1-4}
		\textbf{Parameter} & \textbf{Type} & \textbf{Values} &   \textbf{Default}& \\ \cline{1-4}
		alive & bool & true/false & true \\ \cline{1-4}
        finished & bool & true/false & false \\ \cline{1-4}
	
	\end{tabular}
	\caption{IP Tunnel state variables}
	\label{table:iptunnel_st_var}
\end{table}

\subsection*{Methods}
%
IPTunnel(vector$<$Signal *$>$ \&InputSig, vector$<$Signal *$>$ \&OutputSig)
\bigbreak
bool server(Signal *)
\bigbreak
bool client()


\subsection*{Functional Description}

The IP Tunnel block transmits all elements contained in the signal passed as input. After being executed the input signal's buffer will be empty. This block is duplicated onto two machines, one with input and other with output signal.
