\clearpage

\section{Matched Filter}

\begin{tcolorbox}	
	\begin{tabular}{p{2.75cm} p{0.2cm} p{10.5cm}} 	
		\textbf{Header File}   &:& matched\_filter.h \\
		\textbf{Source File}   &:& matched\_filter.cpp \\
        \textbf{Version}       &:& 20190205 (Daniel Pereira)\\
	\end{tabular}
\end{tcolorbox}

This block applies a matched filter to the input signal.

\subsection*{Input Parameters}

\begin{table}[h]
	\centering
	\begin{tabular}{|c|c|p{60mm}|c|ccc}
		\cline{1-4}
		\textbf{Parameter} & \textbf{Type}   & \textbf{Values} & \textbf{Default} \\ \cline{1-4}
		numberOfTaps       & int             & any             & 64               \\ \cline{1-4}
		rollOffFactor      & double          & any             & 0.5              \\ \cline{1-4}
	\end{tabular}
	\caption{Matched filter input parameters} 
	\label{table:MatchedFilter_in_par}
\end{table}

\subsection*{Methods}

\bigbreak
MatchedFilter(initializer\_list$<$Signal *$>$ \&InputSig, initializer\_list$<$Signal *$>$ \&OutputSig) :Block(InputSig, OutputSig)\{\};
\bigbreak
void initialize(void);
\bigbreak
bool runBlock(void);
\bigbreak
void setNumberOfTaps(int numberOfTaps);
\bigbreak
void setRollOffFactor(double rollOffFactor);

\subsection*{Functional description}


This block implements a matched filter, that coincides with the modulation applied to the signal at the transmitter stage. The filter is applied in frequency domain, where it takes the form of a simple multiplication.

\subsection*{Input Signals}

\textbf{Number:} 1\\
\textbf{Type:} Complex signal

\subsection*{Output Signals}

\textbf{Number:} 1\\
\textbf{Type:} Complex signal

