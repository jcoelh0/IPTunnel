\clearpage

\section{Sink}

\begin{tcolorbox}	
	\begin{tabular}{p{2.75cm} p{0.2cm} p{10.5cm}} 	
		\textbf{Header File}   &:& sink\_*.h \\
		\textbf{Source File}   &:& sink\_*.cpp \\
        \textbf{Version}       &:& 20180523 (Andr\'e Mourato)
	\end{tabular}
\end{tcolorbox}

This block accepts one input signal and it does not produce output signals. It takes samples out of the buffer until the buffer is empty. It has the option of displaying the number of samples still available.

\subsection*{Input Parameters}

\begin{table}[h]
	\centering
	\begin{tabular}{|c|c|p{30mm}|c|ccp{60mm}}
		\cline{1-4}
		\textbf{Parameter} & \textbf{Type} & \textbf{Values} &   \textbf{Default}& \\ \cline{1-4}
		numberOfSamples & long int & any & $-1$ \\ \cline{1-4}
        displayNumberOfSamples & bool & true/false & true \\ \cline{1-4}
	\end{tabular}
	\caption{Sink input parameters}
	\label{table:sink_in_par}
\end{table}

\subsection*{Methods}
%
Sink(vector$<$Signal *$>$ \&InputSig, vector$<$Signal *$>$ \&OutputSig)
\bigbreak
bool runBlock(void)
\bigbreak
void setAsciiFilePath(string newName)
\bigbreak
string getAsciiFilePath()
\bigbreak
void setNumberOfBitsToSkipBeforeSave(long int newValue)
\bigbreak
long int getNumberOfBitsToSkipBeforeSave()
\bigbreak
void setNumberOfBytesToSaveInFile(long int newValue)
\bigbreak
long int getNumberOfBytesToSaveInFile()
\bigbreak
void setNumberOfSamples(long int nOfSamples)
\bigbreak
long int getNumberOfSamples const()
\bigbreak
void setDisplayNumberOfSamples(bool opt)
\bigbreak
bool getDisplayNumberOfSamples const()


\subsection*{Functional Description}

The Sink block discards all elements contained in the signal passed as input. After being executed the input signal's buffer will be empty.
