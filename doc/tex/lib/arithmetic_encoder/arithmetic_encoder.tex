\clearpage

\section{Arithmetic Encoder}

\begin{tcolorbox}	
	\begin{tabular}{p{2.75cm} p{0.2cm} p{10.5cm}} 	
		\textbf{Header File}   &:& arithmetic\_encoder.h \\
		\textbf{Source File}   &:& arithmetic\_encoder.cpp \\
        \textbf{Version}       &:& 20180719 (Diogo Barros) \\
	\end{tabular}
\end{tcolorbox}

This block implements the integer version of the arithmetic encoding algorithm, given the symbol counts, the number of bits per symbol and the number of symbols to encode.
The block takes a binary input stream and outputs the encoded binary stream.

\subsection*{Input Parameters}

\begin{table}[h]
	\centering
	\begin{tabular}{|c|c|c|c|cccc}
		\cline{1-4}
		\textbf{Parameter} & \textbf{Type} & \textbf{Values} & \textbf{Default} & \\ \cline{1-5}
        SeqLen 	   & unsigned int 		  & any & $--$ 	\\ \cline{1-5}
		BitsPerSymb& unsigned int         & any & $--$ 	\\ \cline{1-5}	
	    SymbCounts & vector<unsigned int> & any & $--$  \\ \cline{1-5}	
	\end{tabular}
	\caption{Arithmetic encoder block input parameters.}
	\label{table:arith_enc_in_par}
\end{table}

\subsection*{Methods}

	bool runBlock(void)
\bigbreak
void initialize(void);
\bigbreak
void init(const unsigned int\& SeqLen, const unsigned int\& BitsPerSymb,
const vector<unsigned int>\& SymbCounts);
\bigbreak


\subsection*{Functional description}

This block implements the integer version of the arithmetic encoding algorithm, given the symbol counts, the number of bits per symbol and the number of symbols to encode.


\pagebreak
\subsection*{Input Signals}

\subparagraph*{Number:} 1

\subsection*{Output Signals}

\subparagraph*{Number:} 1

\subparagraph*{Type:} binary

\subsection*{Examples}

\subsection*{Sugestions for future improvement}


