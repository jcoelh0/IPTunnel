\clearpage

\section{Huffman Encoder}

\begin{tcolorbox}	
\begin{tabular}{p{2.75cm} p{0.2cm} p{10.5cm}} 	
\textbf{Header File}   &:& huffman\_encoder\_*.h \\
\textbf{Source File}   &:& huffman\_encoder\_*.cpp \\
\textbf{Version}       &:& 20180621 (MarinaJordao)
\end{tabular}
\end{tcolorbox}

\subsection*{Input Parameters}

The block accepts one input signal,a binary signal with the message to encode, and it produces an output signal (message encoded). 
Two inputs are required, the probabilityOfZero and the sourceOrder.

\begin{table}[h]
	\centering
	\begin{tabular}{|c|c|p{60mm}|c|ccp{60mm}}
		\cline{1-4}
		\textbf{Parameter} & \textbf{Type} & \textbf{Values} &   \textbf{Default}& \\ \cline{1-4}
		probabilityOfZero & double & from 1 to 0 & $0.45$ \\ \cline{1-4}
		sourceOrder & int & 2, 3 or 4 & $2$ \\ \cline{1-4}
	\end{tabular}
	\caption{Huffman Encoder input parameters}
	\label{table:sink_in_par}
\end{table}


\subsection*{Functional Description}


This block encodes a message using Huffman method for a source order of 2, 3 and 4. 

\subsection*{Input Signals}

\textbf{Number}: 1\\
\textbf{Type}: Binary 

\subsection*{Output Signals}

\textbf{Number}: 1\\
\textbf{Type}: Binary
%\end{document}
